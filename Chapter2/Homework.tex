\documentclass{mynote}

\title{Exercise2}
\author{202005100214}
\date\today

\newcommand{\lP}{\mathrm{P}}
% \usepackage{color}

\begin{document}
\maketitle



\begin{exercise}{例三}
    半径为$R_0$的接地导体球置于均匀外电场$\bm{E}_0$中,求电势和导体上的电荷面密度。
\end{exercise}
\begin{solution}
    $r$是球坐标下坐标$(r, \theta, \phi)$的分量。定义球面内势场$u_1$,球面外势场$u_2$。其$u_1$只在球面内有定义,$u_2$只在球面外有定义,两者不能直接叠加。它们分别满足
    \[
    \left\{
        \begin{aligned}
            & \nabla ^2 u_1 = 0 \\
            & u_1 |_{r=R_0} = 0 \\
            & u_1(0) < \infty
        \end{aligned} 
    \right.    
    ,\quad 
    \left\{
        \begin{aligned}
            & \nabla ^2 u_2 = 0 \\
            & u_2|_{r=R_0} = 0 \\
            & \lim_{r\to \infty} u_2 = u_0 - E_0 r P_1(\cos \theta)
        \end{aligned} 
    \right.
    \]
    根据方位角无关的通解公式得到$u_1 = \displaystyle\sum_{n=0}a_nr^nP_n(\cos \theta),\; u_2 = u_0 + \dfrac{b_0}{r} - E_0 r \cos \theta + \dfrac{b_1}{r^2} \cos \theta + \displaystyle\sum_{n=2} \dfrac{b_n}{r^{n+1}}P_n(\cos \theta)$,根据边界条件列出方程
    \begin{align*}
        & \sum_{n=0}a_n R_0^n P_n(\cos \theta) = 0 \\
        & u_0 + \dfrac{b_0}{R_0} - E_0 R_0 \cos \theta + \dfrac{b_1}{R_0^2} \cos \theta + \sum_{n=2} \dfrac{b_n}{R_0^{n+1}}P_n(\cos \theta) = 0
    \end{align*}
    \textbf{由于基底$P_l(x)$的正交性},$\sum_{n=0}a_n R_0^n P_n(\cos \theta) = 0$唯一的情况是$a_n = 0$,则$u_1 = 0$。同时还要求函数在基底$P_n(x)$上的展开系数唯一,即
    \[
    \left\{
        \begin{aligned}
            & u_0 + \dfrac{b_0}{R_0} = 0 \\
            & - E_0 R_0 \cos \theta + \dfrac{b_1}{R_0^2} \cos \theta = 0 \\
            & b_n = 0 \; (n\geq 2)
        \end{aligned} 
    \right.    
    \]
    所以
    \[
    u_2 = u_0 - \dfrac{R_0 u_0}{r} - E_0 r \cos \theta + \dfrac{E_0 R_0^3}{r^2} \cos \theta    
    \]
    如果取$u_0=0$,得到$u_2 = - E_0 r \cos \theta + \dfrac{E_0 R_0^3}{r^2} \cos \theta$。

    球坐标系下的梯度算符为$\nabla = \pp{}{r} \hbm{e}_r + \dfrac{1}{r}\pp{}{\theta} \hbm{e}_{\theta} + \dfrac{1}{r\sin \theta} \pp{}{\phi} \hbm{e}_{\phi}$,球面上$\e_n$大小与方向和$\hbm{e}_r$一致,所以$\pp{u}{n} = \nabla u \cdot \e_n = \pp{u}{r}$.
    带入导体的边界条件$\varepsilon_0 \left. \pp{u}{n} \right|_{r=R_0} = -\sigma$,可以求得约束电荷面密度
    \[
        \sigma = 3 \varepsilon_0 E_0 \cos \theta
    \]
\end{solution}









\begin{exercise}{2.1}
    一个半径为$R$的电介质球,极化强度为$\bm{P} = K \dfrac{\bm{r}}{r^2}$,电容率为$\varepsilon$ \\
    (1)计算束缚电荷体密度和面密度 \\
    (2)计算自由电荷体密度\\
    (3)计算球外和球内电势\\
    (4)求该电介质求的静电场的总能量
\end{exercise}
\begin{solution}
(1)极化强度满足$\nabla \cdot \bm{P} = - \rho_P $,所以
\[
\rho_P = \nabla \cdot K \frac{\bm{r}}{r^2} = -K \frac{{\delta^{\mu}}_{\mu} r^2 - 2r^{\mu}r_{\mu}}{r^4} = -K \frac{1}{r^2}    
\]
球面上取一易拉罐,满足$\e_n \cdot (\bm{P}_2 - \bm{P}_1) = -\sigma_P$,但$\bm{P}_2 = 0$,所以$\sigma_p = P_1 = K \dfrac{1}{R}$

(2)考虑$\bm{D} = \varepsilon_0 + K \dfrac{\bm{r}}{r^2}$以及$\bm{D} = \varepsilon \bm{E}$,得$\bm{D} = \dfrac{\varepsilon K}{\varepsilon - \varepsilon_0} \dfrac{\bm{r}}{r^2}$,由麦克斯韦方程组$\nabla \cdot \bm{D} = \rho$可得
\[
\rho = \dfrac{\varepsilon K}{\varepsilon - \varepsilon_0}  \nabla \cdot \dfrac{\bm{r}}{r^2} = \dfrac{\varepsilon K}{\varepsilon - \varepsilon_0} \frac{1}{r^2}
\]

(3)总的电荷量为
\[
Q = \int_{\mlV} \rho(\bm{r}) \dl \tau = \dfrac{\varepsilon K}{\varepsilon - \varepsilon_0} \int_{0}^{2 \pi} \int_{0}^{\pi} \int_{0}^{R} \frac{1}{r^2} r^2 \sin \theta  \dl r \dl \theta \dl \phi = \frac{4\pi \varepsilon K R}{\varepsilon - \varepsilon_0}
\]
根据高斯定理,球面外的电场满足$E_o(r) 4 \pi r^2 = \dfrac{4\pi \varepsilon K R}{\varepsilon_0 (\varepsilon - \varepsilon_0)}$,解得$E_o(r) = \dfrac{\varepsilon K R}{\varepsilon_0(\varepsilon - \varepsilon_0) r^2}$,从而
\[
u_o(r) = -\int_{\infty}^r \dfrac{\varepsilon K R}{\varepsilon_0(\varepsilon - \varepsilon_0) t^2} \dl t = \dfrac{\varepsilon K R}{\varepsilon_0(\varepsilon - \varepsilon_0) r}
\]
上问已经求得了球内电场$E_i(r) = \dfrac{\varepsilon K}{\varepsilon - \varepsilon_0} \dfrac{1}{r}$,因此球内电势为
\begin{align*}
    u_i(r) &= -\int_{\infty}^{r} E \dl t\\
&= \int_{r}^{R} E_i(t) \dl t + \int_{R}^{\infty} E_o(t) \dl t \\
&= \frac{K}{\varepsilon - \varepsilon_0} \left[ \ln \frac{R}{r} + \frac{\varepsilon}{\varepsilon_0} \right]
\end{align*}

(4) 
\begin{align*}
    W &= \frac{1}{2} \int_{\mlV} \rho(r) u_i(r) \dl r \\
    &= \frac{\varepsilon K^2}{2(\varepsilon - \varepsilon_0)^2} \left[ \iiint_{\mlV} \ln \frac{R}{r} \sin \theta \dl r \dl \theta \dl \phi + \iiint_{\mlV} \frac{\varepsilon}{\varepsilon_0} \sin \theta \dl r \dl \theta \dl \phi  \right] \\
    &= 2\pi \varepsilon R \frac{K^2}{(\varepsilon - \varepsilon_0)^2} \left( 1 + \frac{\varepsilon}{\varepsilon_0}\right)
\end{align*}
\end{solution}






\begin{exercise}{2.2}
    均匀外置电场放入半径为$R_0$的导体球,用分离变量法求解下述情况\\
    (1)导体球接有电池,与地面保持电势差$\varPhi_0$ \\
    (2)导体球带有电荷$Q$
\end{exercise}
\begin{solution}
    (1)球面内定义电势场$u_1$,球面外定义电势场$u_2$。
    两个标量场分别满足
    \[
    \left\{
        \begin{aligned}
            & \nabla^2 u_2(\bm{r}) = 0\\
            & u_2 |_{r=R_0} = \varPhi_0 \\
            & \lim_{r\to \infty} u_2 = u_0 - E_0 r \cos \theta
        \end{aligned} 
    \right. ,\quad
    \left\{
        \begin{aligned}
            & \nabla^2 u_1(\bm{r}) = 0\\
            & u_1 |_{r=R_0} = \varPhi_0 \\
            & u_1(0) < \infty
        \end{aligned} 
    \right.   
    \]
    根据方位角无关的通解公式$u_1 = \displaystyle\sum_{n=0} \left( a_n r^n + \dfrac{b_n}{r^{n+1}} \right) \lP_n(\cos \theta),\; u_2 = \displaystyle\sum_{n=0} \left( c_n r^n + \dfrac{d_n}{r^{n+1}} \right) \lP_n(\cos \theta)$,列出必要条件:$a_0 = u_0,\; a_1 = -E_0,\; a_n = 0\, (n \geq 2),\; d_n = 0$;
    \begin{align*}
        & u_0 + \frac{b_0}{R_0} - E_0 R_0 \cos \theta + \frac{b_1}{R_0^2} \cos \theta + \sum_{n=2} \dfrac{b_n}{R_0^{n+1}} \lP_n(\cos \theta) = c_0 + c_1 R_0 \cos \theta + \sum_{n=2} c_n R_0^n \lP_n(\cos \theta) \\
        & u_0 + \frac{b_0}{R_0} - E_0 R_0 \cos \theta + \frac{b_1}{R_0^2} \cos \theta + \sum_{n=2} \dfrac{b_n}{R_0^{n+1}} \lP_n(\cos \theta) = \varPhi_0
    \end{align*}
    解上述方程带有猜的成分,猜测$b_n = c_n = 0\; (n\geq 2)$,如果有解,则这个解就是唯一的解。
    \[
        u_0 + \frac{b_0}{R_0} - E_0 R_0 \cos \theta + \frac{b_1}{R_0^2} \cos \theta = \varPhi_0
    \]
    为了使上式左边与$\theta$无关,可以让$ - E_0 R_0 \cos \theta + \frac{b_1}{R_0^2} \cos \theta = 0$,得出如下方程组
    \[
    \left\{
        \begin{aligned}
            & E_0 R_0 \cos \theta = \frac{b_1}{R_0^2} \cos \theta \\
            & u_0 + \frac{b_0}{R_0} = \varPhi_0 \\
            & c_0 = u_0 + \frac{b_0}{R_0} \\
            & c_1 = - E_0 R_0 \cos \theta + \frac{b_1}{R_0^2} \cos \theta
        \end{aligned} 
    \right.    
    \]
    发现有解,所以就有
    \[
    u_2(\bm{r}) = u_0 - E_0 r \cos \theta + \frac{R_0 (\varPhi_0 - u_0)}{r} + \frac{E_0 R_0^3}{r^2} \cos \theta
    \]

    (2)根据高斯定理得
    \[
    \oint_{\mlS} \bm{E} \cdot \dl \bm{a} = \frac{Q}{\varepsilon_0}    
    \]
    球面上得单位法向量$\e_n$与球坐标的基失$\hbm{e}_r$是一致的,所以
    \[
        \oint_{\mlS} \bm{E} \cdot \dl \bm{a} = -\oint_{\mlS} \nabla u \cdot \e_n \dl a = -\oint_{\mlS} \pp{u}{r} \dl a
    \]
    在球面上积分,得出边界条件之一$\left. \pp{u}{r} \right|_{r=R_0} = -\dfrac{Q}{4\pi \varepsilon_0 R_0^2}$.
    两个标量场分别满足
    \[
    \left\{
        \begin{aligned}
            & \nabla^2 u_2(\bm{r}) = 0\\
            & \left. \pp{u}{r} \right|_{r=R_0} =  -\dfrac{Q}{4\pi \varepsilon_0 R_0^2} \\
            & \lim_{r\to \infty} u_2 = u_0 - E_0 r \cos \theta
        \end{aligned} 
    \right. ,\quad
    \left\{
        \begin{aligned}
            & \nabla^2 u_1(\bm{r}) = 0\\
            & u_1 |_{r=R_0} = u_2 |_{r = R_0} = \Const \\
            & u_1(0) < \infty
        \end{aligned} 
    \right.   
    \]
    同样的方法解出$b_0 = \dfrac{Q}{4\pi \varepsilon_0},\; b_1 = \dfrac{E_0 R_0^3}{2}$
    \[
        u_2(\bm{r}) = u_0 - E_0 r \cos \theta + \dfrac{Q}{4\pi \varepsilon_0 r} + \dfrac{E_0 R_0^3}{2r^2} \cos \theta
    \]
\end{solution}






\begin{exercise}{2.3}
    均匀介质球的中心置一电荷$Q_f$,球的电容率为$\varepsilon$,球外为真空,用分离变量法球空间电势。
\end{exercise}
\begin{solution}
    定义球面内势场$u_1$,球面外势场$u_2$。根据电介质分界面的边界条件列出方程组
    \[
        \left\{
        \begin{aligned}
            & \nabla^2 u_2 = 0 \\
            & \lim_{r \to \infty} u_2 = 0
        \end{aligned} 
    \right.  
     ,\quad 
     \left\{
        \begin{aligned}
            & \nabla^2 u_1 = - \frac{Q_f\delta(r)}{\varepsilon} \\
            & u_2|_{R_0} = u_1|_{R_0} \\
            & \varepsilon_0 \left. \pp{u_2}{r} \right|_{R_0} = \varepsilon \left. \pp{u_1}{r} \right|_{R_0} \\
        \end{aligned} 
    \right.
    \]
    由于$u_2(\bm{r})$只与长度$r$有关,其通解为$u_2 = a + \dfrac{b}{r}$,根据条件还有$a = 0$。可见在球面上$u_2 = \dfrac{b}{R_0}$是一个常数。由此可以通过
    \[
        \left\{
            \begin{aligned}
                & \nabla^2 u_1 = - \frac{Q_f\delta(r)}{\varepsilon} \\
                & u_1|_{R_0} = \Const
            \end{aligned} 
        \right.
    \]
    确定$u_1$的一族解$u_1 = \dfrac{Q_f}{4\pi \varepsilon r} + c$。通过$ \varepsilon_0 \left. \pp{u_2}{r} \right|_{R_0} = \varepsilon \left. \pp{u_1}{r} \right|_{R_0}$求得$b = \dfrac{Q_f}{4\pi \varepsilon_0}$,通过$u_2|_{R_0} = u_1|_{R_0}$求出$c = \dfrac{Q_f}{4\pi \varepsilon_0 R_0} - \dfrac{Q_f}{4\pi \varepsilon R_0}$。最终得到
    \begin{align*}
        & u_1 = \dfrac{Q_f}{4\pi \varepsilon r} + \dfrac{Q_f}{4\pi \varepsilon_0 R_0} - \dfrac{Q_f}{4\pi \varepsilon R_0} \\
        & u_2 = \dfrac{Q_f}{4\pi \varepsilon_0 r }
    \end{align*}

    % 还有一种等效法,定义在全域上由自由电荷$Q_f$生成的电势$u_f$,显然$u_f = \dfrac{Q_f}{4\pi \varepsilon r}$。定义介质球内极化产生的电势$u_p^i$
\end{solution}







\begin{exercise}{2.6}
    均匀外电场$E_0$置入一均匀带自由电荷密度$\rho_f$的绝缘介质球,电容率为$\varepsilon$,求空间各点电势。
\end{exercise}
\textcolor{red}{外电场不会影响自由电荷分布吗?}
\begin{solution}
    假设外电场下自由电荷分布不变。设球内电势为$u_i + u'_i$,$u_i$是自由电荷产生的电势,$u'_i$是外电场激发的束缚电荷产生的电势;球外电势为$u_e$。分别满足
    \[
    \left\{
        \begin{aligned}
            & \nabla^2 u_i = -\dfrac{\rho_f}{\varepsilon} \\
            & \nabla^2 u'_i = 0 \\
            & [u_i + u'_i]_{r=0} < \infty
        \end{aligned} 
    \right. 
    ,\quad 
    \left\{
        \begin{aligned}
            & \nabla^2 u_e = 0 \\
            & \lim_{r \to \infty} u_e = u_0 - E_0 r P_1(\cos \theta) \\
            & u_e|_{R_0} = [u_i + u'_i ]_{R_0} \\
            & \varepsilon \left. \pp{(u_i + u'_i)}{r} \right|_{R_0}= \left. \varepsilon_0 \pp{u_e}{r} \right|_{R_0}
        \end{aligned} 
    \right.
    \]
    $\nabla^2 u_i = -\dfrac{\rho_f}{\varepsilon}$已经包含了自身电场的极化,无需再考虑别的。而有外电场的情况下会有额外的极化,所以加上了$u'_i$这一项。
    \[
        \nabla^2 u_i = -\dfrac{\rho_f}{\varepsilon} \Rightarrow \nabla \cdot \nabla u_i = - \dfrac{\rho_f}{\varepsilon} \Rightarrow \oint_{\mlS} \pp{u_i}{r} \dl a = -\int_{\mlV} \dfrac{\rho_f}{\varepsilon} \dl \tau    
    \]
    解得$u_i = u_0 - \dfrac{\rho_f}{6\varepsilon} r^2$。目测可知
    \[
        u_2 = u_0 + \frac{b_0}{r} - E_0rP_1(\cos \theta) + \frac{b_1}{r^2}P_1(\cos \theta) + \sum_{n=2} \dfrac{b_n}{r^{n+1}} P_n(\cos \theta)
    \] 
    $u_i + u'_i$形如
    \[
        u_i + u'_i = u_0 - \dfrac{\rho_f}{6\varepsilon} r^2 + \sum_{n=0} c_n r^n P_n(\cos \theta)
    \]
    带入边界条件得
    \begin{align*}
        & u_0 + \frac{b_0}{R_0} - E_0 R_0 P_1(\cos \theta) + \frac{b_1}{R_0^2}P_1(\cos \theta) + \sum_{n=2} \dfrac{b_n}{R_0^{n+1}}  P_n(\cos \theta) \\
        =& u_0 - \dfrac{\rho_f}{6\varepsilon} R_0^2 + c_0 + c_1R_0P_1(\cos \theta) + \sum_{n=2} c_n R_0^n P_n(\cos \theta)
    \end{align*}
    \begin{align*}
        & \varepsilon \left[ -\frac{\rho_f R_0}{3 \varepsilon} + c_1 P_1(\cos \theta) +  \sum_{n=2} n c_n R_0^{n-1} P_n(\cos \theta)\right] \\
        =& \varepsilon_0 \left[ -\frac{b_0}{R_0^2} - E_0 P_1 (\cos \theta)   - \frac{2b_1}{R_0^3} P_1(\cos \theta) + \sum_{n=2} \dfrac{-(n+1) b_n}{R_0^{n+2}}  P_n(\cos \theta) \right]
    \end{align*}
    列出方程组
    \[
    \left\{
        \begin{aligned}
            &  u_0 + \frac{b_0}{R_0} = u_0 - \dfrac{\rho_f}{6\varepsilon} R_0^2 + c_0  \\
            & - E_0 R_0 + \frac{b_1}{R_0^2} = c_1R_0 \\
            & -\frac{\rho_f R_0}{3} = -\frac{\varepsilon_0 b_0}{R_0^2} \\
            & \varepsilon c_1 = \varepsilon_0 \left[ - E_0  - \frac{2b_1}{R_0^3}\right] \\
            & \sum_{n=2} \dfrac{b_n}{R_0^{n+1}}  P_n(\cos \theta) = \sum_{n=2} c_n R_0^n P_n(\cos \theta) \\
            & \varepsilon \sum_{n=2} n c_n R_0^{n-1} P_n(\cos \theta) = \varepsilon_0 \sum_{n=2} \dfrac{-(n+1) b_n}{R_0^{n+2}}  P_n(\cos \theta)
        \end{aligned} 
    \right.    
    \]
    解得$c_n = b_n = 0,\; (n \geq 2)$,
    $b_0 = \dfrac{\rho_f R_0^3}{3 \varepsilon_0}$,
    $c_0 = \dfrac{\rho_f R_0^2}{3 \varepsilon_0} + \dfrac{\rho_f R_0^2}{6 \varepsilon}$,
    $b_1 = \dfrac{-3 E_0 \varepsilon_0 R_0^3}{\varepsilon + 2 \varepsilon_0} + E_0 R_0^3$,
    $c_1 = \dfrac{-3E_0 \varepsilon_0}{\varepsilon + 2 \varepsilon_0}$。所以 
    \begin{gather*}
        u_1 = u_0 - \dfrac{\rho_f}{6\varepsilon} r^2 +  \dfrac{\rho_f R_0^2}{3 \varepsilon_0} + \dfrac{\rho_f R_0^2}{6 \varepsilon} + \dfrac{-3E_0 \varepsilon_0}{\varepsilon + 2 \varepsilon_0} r \cos \theta \\
        u_2 = u_0 + \frac{\rho_f R_0^3 }{3 \varepsilon_0 r} - E_0r \cos \theta + \left(  \dfrac{-3 E_0 \varepsilon_0 R_0^3}{\varepsilon + 2 \varepsilon_0} + E_0 R_0^3 \right)\frac{1}{r^2} \cos \theta
    \end{gather*}
\end{solution}







\begin{exercise}{2.9}
    接地空心导体球内外半径分别为$R_1,\; R_2$,在离球心为$a$处置一电荷$Q$,用镜像法求电势,导体球上感应电荷有多少,分布在内表面还是外表面。
\end{exercise}
\begin{solution}
    感应电荷集中于内表面,因此在等效时也应该以内表面为准。由相似关系得$R_1^2 = ab$,在$R_1$面上还应满足$\dfrac{Q}{4\pi \varepsilon_0 r_0} = \dfrac{q}{4\pi \varepsilon_0 r_1}$,得$q = \dfrac{r_1}{r_0} = \dfrac{R_1}{a}$,所以
    \[
    u(r) = \dfrac{Q}{4\pi \varepsilon_0 r_0} - \dfrac{QR_1}{a 4 \pi \varepsilon_0 r_1}
    \]
    同时用余弦定理计算$r_0 = \sqrt{r^2 + a^2 - 2ra\cos \theta}$,$r_1 = \sqrt{r^2 + \dfrac{R_1^4}{a^2} - 2r\dfrac{R_1^2}{a} \cos \theta}$,故
    \[
    u(r) = \dfrac{Q}{4\pi \varepsilon_0 \sqrt{r^2 + a^2 - 2ra\cos \theta}} - \dfrac{QR_1}{a 4 \pi \varepsilon_0 \sqrt{r^2 + \dfrac{R_1^4}{a^2} - 2r\dfrac{R_1^2}{a} \cos \theta}}
    \]
\end{solution}





\begin{exercise}{2.10}
    上题导体球壳不接地,而是带总电荷$Q_0$,或使其有总电荷$\varPhi_0$,求其电势。
\end{exercise}
\begin{solution}
    设球壳内电势$u_1$,球壳外电势$u_2$,满足
    \[
    \left\{
        \begin{aligned}
            & \nabla ^2 u_2 = 0 \\
            & \lim_{r \to \infty} u_2 = 0 \\
            & \oint_{r=R_2} \pp{u_2}{r} \dl a = - \dfrac{Q + Q_0}{\varepsilon_0} 
        \end{aligned} 
    \right.    
    ,\quad 
    \left\{
        \begin{aligned}
            & \nabla^2 u_1 = - Q \dfrac{\delta(\bm{r} - \bm{a})}{\varepsilon_0} \\
            & u_1 |_{R_1} = u_2|_{R_2} = \Const
        \end{aligned} 
    \right.
    \]
    可以直接解出$u_2 = \dfrac{Q + Q_0}{4\pi \varepsilon_0 r}$。$u_1$的解形如
    \[
    u_1 = \dfrac{Q}{4\pi \varepsilon_0 \sqrt{r^2 + a^2 - 2ra\cos \theta}} - \dfrac{QR_1}{a 4 \pi \varepsilon_0 \sqrt{r^2 + \dfrac{R_1^4}{a^2} - 2r\dfrac{R_1^2}{a} \cos \theta}} + c
    \]
    带入边界条件$u_1 |_{R_1} = \dfrac{Q + Q_0}{4\pi \varepsilon_0 R_2}$,有
    \begin{align*}
        &\dfrac{Q}{4\pi \varepsilon_0 \sqrt{R_1^2 + a^2 - 2R_1a\cos \theta}} - \dfrac{QR_1}{a 4 \pi \varepsilon_0 \sqrt{R_1^2 + \dfrac{R_1^4}{a^2} - 2R_1\dfrac{R_1^2}{a} \cos \theta}} + c = \dfrac{Q + Q_0}{4\pi \varepsilon_0 R_2}  \\
        & \Rightarrow \dfrac{Q}{4\pi \varepsilon_0 \sqrt{R_1^2 + a^2 - 2R_1a\cos \theta}} - \dfrac{Q}{4\pi \varepsilon_0 \sqrt{R_1^2 + a^2 - 2R_1a\cos \theta}} + c = \dfrac{Q + Q_0}{4\pi \varepsilon_0 R_2} \\
        & \Rightarrow c = \dfrac{Q + Q_0}{4\pi \varepsilon_0 R_2}
    \end{align*}
    所以
    \[
    u_1 =     \dfrac{Q}{4\pi \varepsilon_0 \sqrt{r^2 + a^2 - 2ra\cos \theta}} - \dfrac{QR_1}{a 4 \pi \varepsilon_0 \sqrt{r^2 + \dfrac{R_1^4}{a^2} - 2r\dfrac{R_1^2}{a} \cos \theta}} + \dfrac{Q + Q_0}{4\pi \varepsilon_0 R_2}
    \]

    如果只给定导体的电势,也可以仿照上面的方法求出
    \begin{align*}
        & u_1 = \dfrac{Q}{4\pi \varepsilon_0 \sqrt{r^2 + a^2 - 2ra\cos \theta}} - \dfrac{QR_1}{a 4 \pi \varepsilon_0 \sqrt{r^2 + \dfrac{R_1^4}{a^2} - 2r\dfrac{R_1^2}{a} \cos \theta}} + \varPhi_0 \\
        & u_2 = \dfrac{\varPhi_0 R_2}{r}
    \end{align*}
\end{solution}












\begin{exercise}{2.11}
    接地导体平面有一半径为$a$的半凸部,半球球心在导体平面上,点电荷$Q$位于系统的对称轴上,与平面相距为$b$,求空间电势。
\end{exercise}
\begin{solution}
    为使半球面上电势为$0$,可以等效为球面内有一未知电荷$q$。$q$在对称轴上,设其位置半径为$d$,应当满足$bd=a^2$,即$d = \dfrac{a^2}{b}$,其与电荷$Q$的势能叠加为$0$,即$\dfrac{q}{4\pi \varepsilon_0 r_a} + \dfrac{Q}{4\pi \varepsilon_0 r_b} = 0 \Rightarrow q = -\dfrac{r_a}{r_b} Q \Rightarrow q = -\dfrac{a}{b}Q$.此时平面的电势还不是$0$,由于上述的电荷结构对球面的电势没有影响,所以考虑在平面下方对称地放置两个电荷,位置分别为$-\dfrac{a^2}{b},\; -b$,电荷量分别为$\dfrac{a}{b}Q,\; -Q$。最终空间中某处的电势可以表示为
    \begin{align*}
        u(\bm{r}) = \dfrac{Q}{4\pi \varepsilon_0} \left( \dfrac{1}{\sqrt{x^2 + (y-b)^2}} - \dfrac{1}{\sqrt{x^2 + (y+b)^2}} \right) + \dfrac{q}{4\pi \varepsilon_0} \left( \dfrac{-1}{\sqrt{x^2 + (y-\frac{a^2}{b})^2}} + \dfrac{1}{\sqrt{x^2 + (y+\frac{a^2}{b})^2}} \right)
    \end{align*}
\end{solution}




















\end{document}